\documentclass[nobib]{tufte-handout}
\usepackage{amsmath}

\title{INTG 335 \\ Paper \#1 }
%\author{James Logan Mayfield }
\date{ Fall 2017 }

\begin{document}
\maketitle
\thispagestyle{empty}

\begin{center}
  \textbf{Due in class on Monday 9/18}
\end{center}

\subsection*{The Prompt}

The era of human supremacy in games of intelligence and strategy appears to be coming to an end. The classics like Checkers, Chess, and Go have fallen to AI with seemingly superhuman playing capabilities. Games of chance like Poker are beginning to fall as well. Major developers of AI such as DeepMind are having great success in the realm of video games as well. None the less, humans continue to dedicate their lives to the pursuit of excellence in these games.

%% Refine/Edit/Narrow/Focus these questions
In your first paper you will explore how, if at all, the efficacy of such pursuits has changed in a post-AI-supremacy world. Has the pursuit of excellence in games lost or gained value as the result of the rise of the machines? Does it matter that our benchmarks, the players whose skill marks the upper eschlon of the game, are machines and not human beings?  Does the existence of a super-human player and in some cases a perfect player\sidenote{i.e. Chinook and the solution to Checkers} make playing a game well more or less meaningful? More generally, what if anything, does it say about our own intelligence if machines can beat us at our own games?

\section{Requirements \& Logistics}

\begin{tabular}{ll}
\textsc{Due} &  \\
\textsc{Format} & Typed. Printed. Double-spaced. Stapled. \\
\textsc{Length} & 750--1250 words \\
\textsc{Min. Sources} & 3
\end{tabular}
\vspace{.25in}

Grades are based on the quality and clarity of the writing, the appropriateness and quality of observations, the depth and sufficiency of the analysis, and the clarity and validity of the conclusions with respect to the overall reflections. Be sure to review \textit{On Reflection and Reflective Writing}\sidenote{\url{https://jlmayfield.github.io/teaching/INTG335/intg335-reflection.pdf}} if needed.

\section{The Writing Center}

Wonderful ideas and deep reflection can be lost or ruined by muddled writing.  I grew most as a writer when I cared deeply about the things I was writing and didn't want there to be any uncertainty on the part of my readers. Perhaps more than other forms or writing, reflective writing is ultimately a reflection of ourselves. You should want to put your best word forward when it's you on the page. Writing well takes practice. It requires help. I encourage \textit{all} of you to take this opportunity to step up your writing game. The writing center is here to help you. They're awesome. Don't wait until my grade to get feedback on the quality and clarity of your writing.
\begin{quote}
  The Writing Center offers unlimited, free peer tutoring sessions for Monmouth College students.  Peer writing tutors are trained to work with writers from any major, of any writing ability, on any type of writing assignment, and at any stage of their writing processes, from planning to drafting to revising to editing.  Peer speech tutors are also available on a limited basis to assist student speakers at any point in the process of designing a speech – from outlining to delivery.  The Writing Center is  located on the 3rd floor of the Mellinger Teaching and Learning Center, and is open Sunday-Thursday 7--10pm and Monday-Thursday 3--5pm on a first-come, first-served basis.  No appointment necessary!
\end{quote}

\end{document}
