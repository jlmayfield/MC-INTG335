\documentclass[nobib]{tufte-handout}

\title{INTG 388 \\ Paper \#3 }
%\author{ James Logan Mayfield }
\date{ Fall 2017 }

\begin{document}
\maketitle

\section{Prompt}

The AI we explored prior to this section of the course have primarily appealed to our more pragmatic side. They play games better, they do jobs better, and generally optimize some task that we want done along some metric such that there is a clear benefit. That is, these AI save us things like time and money. In this third section of the course we've begun looking at AI that provide for needs and desires of a more emotional nature or that simply invoke some kind of emotional response from us. Frank develops a meaningful friendship with Robot. Chat-bots provide a captive audience ready to engage in conversation.  AI bots can even be created in the likeness of departed love ones. These kinds of applications of AI appear to be of a different kind and merit some reflection.

% Refine
For this paper you'll examine the use of weak AI for tasks that clearly tap into human emotion. Should we be concerned about the mechanization and automation of our emotional selves? Is it OK to have a relationship, of any kind, with a weak AI\@? Can we, in fact, learn something more about being human from the kinds of emotionally charged interactions that we have with our AI or will such interactions somehow weaken human interactions?


\section{Requirements \& Logistics}

\begin{tabular}{ll}
\textsc{Due} &  \\
\textsc{Format} & Typed. Printed. Double-spaced. Stapled. \\
\textsc{Length} & 750--1250 words \\
\textsc{Minimum Required External Sources} & 3 minimum.
\end{tabular}

Grades are based on the quality and clarity of the writing, the appropriateness and quality of observations, the depth and sufficiency of the analysis, and the clarity and validity of the conclusions with respect to the overall reflections. Be sure to review \textit{On Reflection and Reflective Writing}\sidenote{\url{https://jlmayfield.github.io/teaching/INTG335/intg335-reflection.pdf}} if needed.

\section{The Writing Center}

Wonderful ideas and deep reflection can be lost or ruined by muddled writing.  I grew most as a writer when I cared deeply about the things I was writing and didn't want there to be any uncertainty on the part of my readers. Perhaps more than other forms or writing, reflective writing is ultimately a reflection of ourselves. You should want to put your best word forward when it's you on the page. Writing well is hard. It takes practice. I requires help. I encourage \textit{all} of you to take this opportunity to step up your writing game. The writing center is here to help you. They're awesome. Don't wait until my grade to get feedback on the quality and clarity of your writing.
\begin{quote}
The Monmouth College Writing Center offers unlimited, free peer tutoring sessions for students at MC\@.  Peer writing tutors work with writers from any major, of any writing ability, on any type of writing assignment, and at any stage of their writing processes, from planning to drafting to revising to editing.  We are located on the 3rd floor of the Mellinger Teaching and Learning Center, and the Writing Center is open Sun-Thurs 7--10pm, and Mon-Thurs 3--5pm on a first-come, first-served basis.  No appointment necessary!  International students can sign up to work with the same writing tutor for a weekly session. Learn more about the Writing Center at our website: \url{http://blogs.monm.edu/writingatmc/writing-center/}
\end{quote}


\end{document}
