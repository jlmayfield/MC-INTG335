\documentclass[]{tufte-handout}
\usepackage{amsmath}

\title{INTG 388 - Paper \#1 - Intellectual Achievements}
%\author{James Logan Mayfield }
\date{ Spring 2016 }

\begin{document}
\maketitle
\thispagestyle{empty}


\subsection*{The Prompt}

Weak AIs embody the the so called ``curse of artificial intelligence''\sidenote{\url{http://www.artificial-intelligence.com/comic/7}}.  They're machines, human-built tools, that automate an intellectual process thereby turning that process into something that can be generated with the flip of a switch as opposed to some mysterious process that arises from human consciousness. In this paper you are to consider whether or not this curse is really just the ``curse of \textit{intelligence}''?  Put another way, how does increased knowledge, experience, and understanding impact how you value intelligence and intellectual achievement and how does it shape what intelligence means to you? 

In exploring this question you should examine your own intellectual endeavors and achievements, those of other humans, and those of weak AI. Do not wade into the quagmire that is strong AI. Choose sources and observables that give you variance and contrast in and between human and machine intelligence. You might consider how your valuation of the intelligence need for some tasks may have changed over time as your knowledge has grown. You could also explore the quality of the result of intelligent endeavors and whether or not you would be impressed by a weak AI that achieved the equivalent of a human amateur as opposed to the expert level and beyond achievements we've seen in class.  No matter what direction you take, you should focus your reflection on the central theme of the impact of experience, knowledge, and understanding on the valuation of intelligence and how weak AI does or doesn't inform your understanding of this issue. 

\section{Requirements \& Logistics}

\begin{tabular}{ll}
\textsc{Due} & Beginning of class on Wednesday 2/3. \\
\textsc{Format} & Typed. Printed. Double-spaced. Stapled. \\
\textsc{Length} & 750 - 1250 words \\
\textsc{Sources} & minimum of 3 \\
\end{tabular}

Grades are based on the equal weighing of quality and clarity of the writing, the appropriateness and quality of observations, the depth and sufficiency of the analysis, and the clarity and validity of the conclusions with respect to the overall reflections.  You should use the writing center whenever possible and don't forget that there's a whole document discussing the nature and style of reflective writing.


\end{document}