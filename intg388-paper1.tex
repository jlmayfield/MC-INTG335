\documentclass[]{tufte-handout}
\usepackage{amsmath}

\title{INTG 388 - Paper \#1 - Intellectual Achievements}
\author{James Logan Mayfield }
\date{ Spring 2015 }

\begin{document}
\maketitle

\section{Reflective Writing}

Reflection is an inward looking process. We're trying to explore what's inside us as individuals and a species. This paper is not about making a point or presenting information. It's about exploring an experience, and idea, or a possible answers to a questions and seeing how these things impacts us on a personal level. We're exploring the human condition and how we make sense of it. We can think of this process in terms of three phases:
\begin{enumerate}
\item \textbf{Observation:} Students should carefully describe, explain, or notice. Their writing should make clear that they've taken some time to carefully observe. \newline
\item \textbf{Analysis:} After observing, students should then explore. They should break ideas down and put them back together. They should begin making connections between ideas encountered in the course and their own experience. They should consider what their reflection means to them intellectually, personally, and socially. \newline
\item \textbf{Deriving Implications and Meaning:} Students should offer conclusions about where the reflection has brought them. These conclusions may be partial, plural, or not fully settled, but students should be able to articulate how they think differently at the end of their reflection. They may have new ideas. They may have new understandings of old beliefs. In any case, they should articulate where they stand in relation to the reflection practiced. 
\end{enumerate}


\section{Prompts}

We've started this class by exploring the intellectual achievements of machines and seeing how they stack up against human intelligence.  In doing so we hoped to better understand intelligence, both natural and artificial. For this paper, you'll use some specific achievements to reflect on general nature of intellectual achievement and how the achievements of machines change, or don't change, how we view the achievements of humankind. 

To further focus your efforts, you are to use one of these prompts as a launching off point for your reflection:
\begin{enumerate}
\item \textsc{The Producer} 

To what extent is the value of an intellectual achievement influenced by the producer?  Should we value something created by humans more than machines? What about within each group of intelligences? Are there humans who's intellectual results we should hold up more than others or certain types of machines whose products are more valuable than other machines?  Now make it personal. How would you feel if a machine produced the same thing you did? What if the machine were better? What if you found you reinvented something someone else has done or vice versa?  

\item \textsc{The Process} 

To what extent is the intelligence inherent in an achievement a function of the process at play?  Is the so called ``curse of artificial intelligence''\sidenote{\url{http://www.artificial-intelligence.com/comic/7}} really just the ``curse of \textit{intelligence}''?  Do we only deem a process as one involving intelligence if it is mysterious or closed to us?  Can following instructions produce intellectual achievements? Are all rules created equal or do some take intelligence to apply and other not? Now make it personal. How do you apply your intelligence? Is it rule based or does intelligence only come into play when the rules run out? Should you value an intellectual achievement an less if it was achieved by means of someone or something else's guide? 

\item \textsc{Choose Your Own Adventure} 

Propose a prompt that explores the intellectual achievements of humankind and machines. \textit{Self generated prompts must be approved by the instructor at least 1 week prior to the paper's due date.}

\end{enumerate} 

The prompts are not mutually exclusive. You should, however, focus on the core of your prompt and only use issues from the other prompt as supportive evidence and arguments. Similarly, these prompts can lead to issues more appropriately discussed in future papers focused on the remaining course umbrella topics. So, be certain to ground your reflection in our exploration of the intellectual achievements of machines as we've looked at them thus far. Above all else, remember that reflective writing ultimately requires us to turn our critical eye inward and consider how our thinking about the topic at hand shapes who your are, what you're doing, and what you'll do and think in the future. 

\section{Requirements \& Logistics}

\begin{tabular}{ll}
\textsc{Due} & Beginning of class on Friday 2/14. \\
\textsc{Format} & Typed. Printed. Double-spaced. Stapled. \\
\textsc{Length} & 750 - 1250 words 
\end{tabular}

Grades are based on the quality and clarity of the writing, the appropriateness and quality of observations, the depth and sufficiency of the analysis, and the clarity and validity of the conclusions with respect to the overall reflections.  

\section{The Writing Center}

Wonderful ideas and deep reflection can be lost or ruined by muddled writing.  I grew most as a writer when I cared deeply about the things I was writing and didn't want there to be any uncertainty on the part of my readers. Perhaps more so that other forms or writing, reflective writing is ultimately a reflection of ourselves. You should want to put your best word forward when it's you on the page. Writing well is hard. It takes practice. I requires help. I encourage \textit{all} of you to take this opportunity to step up your writing game. The writing center is here to help you. They're awesome. Don't wait until I'm grading your paper to get feedback on the quality and clarity of your writing. 
\begin{quote}
The Monmouth College Writing Center offers unlimited, free peer tutoring sessions for students at MC.  Peer writing tutors work with writers from any major, of any writing ability, on any type of writing assignment, and at any stage of their writing processes, from planning to drafting to revising to editing.  We are located on the 3rd floor of the Mellinger Teaching and Learning Center, and we are open Sunday-Thursday 7-10pm and Monday-Thursday 3-5pm on a first-come, first-served basis.  No appointment necessary!  International students can sign up to work with the same writing tutor for a weekly session. Learn more about the Writing Center at our website: \url{http://blogs.monm.edu/writingatmc/writing-center/}
\end{quote}


\end{document}