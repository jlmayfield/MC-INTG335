\documentclass[nobib]{tufte-handout}
\usepackage{amsmath}

\title{INTG 388 - Paper \#2 }
\author{}
\date{ Spring 2016 }

\begin{document}
\maketitle

\section{Prompt}

We don't need to look into the future to consider the ramifications of whether or not you choose to adopt a piece of AI-enhanced technology\sidenote{if you cannot find something utilizing AI, then you can get approval for non-AI based technology from the instructor} into your life.  The future is now\sidenote{sorry... had to say it}.  

You may approach this paper from one of three directions:
\begin{enumerate}
\item \textit{I would never} \newline
Identify a piece of technology you've made it a point to avoid and consider the ramifications of your choice.

\item \textit{I would totally} \newline
Identify a piece of technology that you've gladly incorporate into your life and consider the ramifications of your choice. 

\item \textit{That's just the way it is} \newline
Identify a piece of technology that you've adopted without any real conscious choice in the matter and consider the ramifications of your lack of oversight. 
\end{enumerate}

Examine the quantity and quality of control surrendered to the  technology as well as the short and long term impact of its adoption. A proper analysis must consider the reasons for and against using the technology. Your reflection should balance those reasons against your own decisions or lack thereof. As our focus is on AI-based and AI-enhanced technology, you should be certain to examine your intellectual life as well as life in general. Above all, be certain your reflection presents and develops a clear statement about how looking at your choices with respect to technology have informed your sense of what's right and wrong. To ground your analysis and reflection, you must seek out at least three external sources. These sources should provide perspective on the technology and the consequences of using it. Your paper must use proper citations and include a bibliography. All sources must be added to the group library on Zotero.  

\newpage \thispagestyle{empty}

\section{Requirements \& Logistics}

\begin{tabular}{ll}
\textsc{Due} & Before you leave for spring break  \\
\textsc{Format} & Typed. Printed. Double-spaced. Stapled. \\
\textsc{Length} & 750 - 1250 words \\
\textsc{Min. Sources} & 3
\end{tabular}
\vspace{.25in}

Grades are based on the quality and clarity of the writing, the appropriateness and quality of observations, the depth and sufficiency of the analysis, and the clarity and validity of the conclusions with respect to the overall reflections. You should re-familiarize yourself with the course \textit{Notes on Reflection}\sidenote{\url{https://jlmayfield.github.io/MC-INTG388/intg388-reflection.pdf}} if needed. 

\section{The Writing Center}

Wonderful ideas and deep reflection can be lost or ruined by muddled writing.  I grew most as a writer when I cared deeply about the things I was writing and didn't want there to be any uncertainty on the part of my readers. Perhaps more than other forms or writing, reflective writing is ultimately a reflection of ourselves. You should want to put your best word forward when it's you on the page. Writing well is hard. It takes practice. I requires help. I encourage \textit{all} of you to take this opportunity to step up your writing game. The writing center is here to help you. They're awesome. Don't wait until my grade to get feedback on the quality and clarity of your writing.  
\begin{quote}
The Monmouth College Writing Center offers unlimited, free peer tutoring sessions for students at MC.  Peer writing tutors work with writers from any major, of any writing ability, on any type of writing assignment, and at any stage of their writing processes, from planning to drafting to revising to editing.  We are located on the 3rd floor of the Mellinger Teaching and Learning Center, and the Writing Center is open Sun-Thurs 7-10pm, and Mon-Thurs 3-5pm on a first-come, first-served basis.  No appointment necessary!  International students can sign up to work with the same writing tutor for a weekly session. Learn more about the Writing Center at our website: \url{http://blogs.monm.edu/writingatmc/writing-center/}
\end{quote}


\end{document}