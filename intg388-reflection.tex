\documentclass[]{article}
\usepackage{amsmath}
\usepackage{setspace}
\usepackage{hyperref}


\setlength{\textheight}{9in} \setlength{\topmargin}{-.5in}
\setlength{\textwidth}{6.5in} \setlength{\oddsidemargin}{0in}
\setlength{\evensidemargin}{0in}


\title{INTG 388 - Notes on Reflective Writing}
\author{James Logan Mayfield }
\date{ Spring 2016 }


\begin{document}
\maketitle
\thispagestyle{empty}

\subsection*{What is Reflection?}

Reflection is an inward looking process. When reflecting on something, we are trying to explore what's inside us as individuals and a species. It's not about making a point or presenting information. It's about exploring an experience, an idea, or a possible answer to a question and seeing how it impacts us on a personal level. We're exploring the human condition and how we make sense of it. We can think of this process in terms of three phases:
\begin{enumerate}
\item \textbf{Observation:} Carefully describe, explain, or notice. Your writing should make clear that you've taken some time to carefully observe a thing. \newline
\item \textbf{Analysis:} After observing, explore. Break ideas down and put them back together. Make connections between ideas encountered in the course and your own experience. Consider what your reflection means to you intellectually, personally, and socially. \newline
\item \textbf{Deriving Implications and Meaning:} Offer conclusions about where the reflection has brought you. These conclusions may be partial, plural, or not fully settled, but you should be able to articulate how you think differently at the end of their reflection. You may have new ideas. You may have new understandings of old beliefs. In any case, You should articulate where you stand in relation to the reflection practiced. 
\end{enumerate}

\subsection*{How to write about Reflection}

The style of reflective writing I advocate begins by treating your conclusion as the ``thesis''. It is thing you're presenting, explaining, and developing in the course of your writing. Like a thesis, it should be clearly stated in the introductory paragraph and then developed and expanded upon in the remainder of the document. Too often I see students carry out the reflection in writing then submit that as their reflective writing.  They observer, then analysis, then conclude.  The end result, from the reader's perspective, is usually confusing. It's not clear where you're going with this and why we should keep reading. You, the reflector, should absolutely observer, analyze, then conclude, but you should do that \textit{before you write the paper or journal entry}.  Writing your way through the reflection isn't a terrible idea, just don't expect to receive good marks on it if it's submitted as is. You're essentially turning in a first draft when you do this. 

Your goal as the writer is to make sure the reader understands the path that lead you to your new way of thinking. This is why you should present the conclusion right up front. The reader will begin their journey with a broad understanding of  where your thinking was before the reflection, the object of your reflection, and the new way of thinking that resulted from the reflection. You then fill on the details and conclude the essay by going back to the conclusion of the reflection with the details fully explored. \textit{You are 100\% not trying to convince them that your new way of thinking is \textit{the} way to think on the subject nor are you informing them of new facts.} The subject the reader should learn the most about from reading your essay is you. Their thinking may change or they may learn some new facts and ideas as a result of reading your essay, but that's secondary to the real goal of presenting and explaining your change in thinking and the reflection that got you there. 




\end{document}