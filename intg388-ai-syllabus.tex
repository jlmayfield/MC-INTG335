\documentclass[]{tufte-handout}
\usepackage{amsmath}

\title{Syllabus: INGT 388 - Artificial Intelligence and the Singularity}
\author{James Logan Mayfield }
\date{ Spring 2016 }

\begin{document}
\maketitle

\section{Logistics}
\begin{itemize}
\item \textbf{Where: } Center for Science and Business, Room 380
\item \textbf{When: } MWF 11:00 - 11:50 am
\item \textbf{Instructor :} James \textit{Logan} Mayfield
\begin{itemize}
\item \textit{Office: } Center for Science and Business, Room 344
\item \textit{Phone: } 309-457-2200
\item \textit{Email: } lmayfield \textit{at} monmouthcollege \textit{dot} edu
\item \textit{Office Hours: } By Appointment*
\end{itemize}
\item \textbf{Credits: } 1 course credit
\end{itemize}

\section{Overview}

Is humanity on the verge of a technological singularity, a moment in time where computational artificial intelligence (AI) exceeds the capabilities of natural human intelligence?  If so, what does that mean for humankind and its future? AI based technologies have already demonstrated their ability to challenge human intelligence at games of skill and knowledge such as checkers, chess and Jeopardy. They've also begun to identify and discover meaningful patterns and information from data in order to perform diagnostic and predictive tasks better than human experts.  What is perhaps more surprising is that AI has also begun to produce works of art comparable to those of human artists. This class explores the myths and realities of intelligent machines and in doing so address questions about the origins, uniqueness, and prominence of human intelligence in the universe.     

The class will explore four broad topics:
\begin{enumerate}
\item Weak AI: Human vs. Machine
\item Weak AI: Human and Machine
\item The Emotional Machine
\item Strong AI and Consciousness
\end{enumerate}

\section{Sources}

In addition to sources chosen by the instructor, students will regularly identify and choose resources for topics that interest them most within the umbrella of AI and the four course topics. These sources will be used for writing assignments and to generate new class discussion.  This means that the source list for the course is dynamic and will grow as the semester progresses. 

The current list of sources is maintained by both students and the instructor as a Zotero group library and is viewable at the following URL:
\vspace{.2in}

\begin{footnotesize}
\url{https://www.zotero.org/groups/monmouth_college_intg388-01_sp16}
\end{footnotesize}

 
\section{Assignments}

The course assignment workload is as follows:
\begin{itemize}
\item Reflective Writing Journal. 
\item Regular participation in class discussions
\item 3 750-1250 word papers
\item 1 1500-2000 word paper
\item 1 Presentation 
\end{itemize}
The four papers will correspond to the four themes of the course. The presentation will take place during the finals period and will give students an opportunity to present something not covered in the class or to go deeper on something that did arise in class.

\subsection{Reflective Writing Journal}

The reflective writing journal is where students explore AI topics and applications not discussed in class or where they can go beyond the in class discussions. Ideally, journal entries act as idea generators for class discussions and the major papers. Each entry must be 250-500 words and should be a cohesive, well formed piece of reflective writing. Entries should not rehash class discussions. Entries should not simply report on something read, viewed, or listened to. When class discussions are the object of consideration and reflection, then students should go above and beyond what was discussed in class. Each entry must utilize one and only one source. That source must be added to Zotero if it's not already there. Within the journal entry itself, students should make proper use of inline citations and provide a complete and correct bibliographic entry for the source. 

Journals will be collected every Monday. Not more then two new journal entries will be accepted in any one given week. Journal entries are graded on a pass or fail basis. Focused, specific, cohesive, well written pieces of reflective writing with proper citations can expect to receive a pass. Entries will receive a failing mark if they lack the necessary components of reflective writing or are overly general and vague. It is better to reflect deeply on something concrete and specific than it is to spread your reflection across something broad and generic. Students can also receive a failing mark if the entry's source isn't well documented on Zotero or is improperly cited in the entry. Finally, a journal entry can fail if the writing itself is so laden with typos and grammatical mistakes as to render it overly difficult to read.  Students should take these short, low-stress writing exercises seriously and use them as an opportunity to develop their skills as a writer. 

The overall journal grade is based on the number of passing journal entries submitted by the end of the semester. 

\begin{center}
\begin{tabular}{ll}
\underline{Grade} & \underline{Number Passed} \\
A (.95) & 16 \\
B (.85) & 14 \\
C (.75) & 12 \\
D (.65) & 10\\
F (.35) & 8 \\
\end{tabular}
\end{center}

To ensure that you spread your journal writing across the whole semester, you cannot rack up more than eight passing journal entries per half semester. You are, however, allowed eight revisions and resubmissions throughout the course of the whole semester. Not more than one revision will be accepted per week. The revision does not count as one of your two new entries. So, in any given week you can submit two new journal entries and one revision. Revisions must be submitted along with the original entry. 


\subsection{Participation \& Attendance}

This class does not work if students are not \textit{all} freely presenting and discussing ideas.  This means each student must be in class, willing to work through and with another student's ideas, and willing to present their own ideas.  \textit{A student is allowed \textbf{one} unexcused absence from class.} Excused absences must be discussed with the instructor prior to the class missed.  Failure to do so is likely to result in an unexcused absence.  Every unexcused absence after the first will result in the loss of $1/5$ of the course participation grade until the participation grade drops to zero after five unexcused absences. This means five or more unexcused absences caps the student's course grade at a maximum of a C. What constitutes an approved, excused, and unexcused absence is well documented in the Scots guide under the college absence policy. 

Assuming good attendance, the overall attendance and participation grade is a function of a student's contributions to in class discussion and to the Zotero library.  Discussions will occur in small groups and as a whole class. Students should contribute in both cases. Every student should make some contributions to the Zotero library.  Failure to do so means you're papers and journal entries do not go beyond material presented by the instructor or by other students.  Good reflection requires a diversity of ideas and perspectives and that requires a wide array of sources. There is not a set grading rubric for these things, but regular feedback and will be given through comments in journal submissions. 

\subsection{Papers}

Students in this class will be  assigned four papers. The first three papers are all 750-1250 words. The final paper is longer: 1500-2000 words. All four papers will require multiple sources and at least some of them must be scholarly. These papers give students the opportunity to tie together multiple journal entries and in-class discussions into a single piece of reflective writing. The final paper will be submitted via Turn-It-In.com as part of the Reflections/INTG assessment process. 

\subsection{The Writing Center}

The Monmouth College Writing Center offers unlimited, free peer tutoring sessions for students at MC.  Peer writing tutors work with writers from any major, of any writing ability, on any type of writing assignment, and at any stage of their writing processes, from planning to drafting to revising to editing.  They are located on the 3rd floor of the Mellinger Teaching and Learning Center.Learn more about the Writing Center at our website: \url{http://blogs.monm.edu/writingatmc/writing-center/}

\section{Grading}

This courses uses a standard grading scale.  Assignments and final grades will not be curved except in rare cases when its deemed necessary by the instruction.  Percentage grades translate to letter grades as follows:
\newline
\begin{center}
\begin{small}
\begin{tabular}{ll}
Score Range & Grade \\ \hline
94-100 & A \\
90-93 & A- \\
88-89 & B+ \\
82-87 & B \\
80-81 & B- \\
78-79 & C+ \\
72-77 & C \\
70-71 & C- \\
68-69 & D+ \\
62-67 & D \\
60-61 & D- \\
0-59 & F 
\end{tabular}
\end{small}
\end{center}


Students are always welcome to challenge a grade that they feel is unfair or calculated incorrectly.  Mistakes made in the student's favor will never be corrected to lower a grade.  Mistakes made not in a student's favor will be corrected.  \textit{Basically, after the initial grading of an assignment, it's score can only go up as the result of a challenge.}


\subsection{Grade Weights}

The final grade is based on a weighted average of particular assignment categories.  Students may visit the instructor \textit{outside of class time} to discuss their current standing.  

\begin{center}
\begin{tabular}{ll}
Category & Weight  \\ \hline
Participation \& Attendance & 25\% \\
Journal & 25\% \\
First Paper & 5\% \\
Second Paper & 10\% \\
Third Paper & 15\% \\
Final Paper & 15\% \\
Presentation & 5\% \\
\end{tabular}
\end{center}


\section{Calendar}

\textit{This calendar is subject to change based on the circumstances of the course.}  The \textit{Journals Remaining} column lists how many more new entries for the current half-semester can be submitted after that week's submission.

\begin{center}
\begin{tabular}{|c|c|r|c|}
\hline 
Week & Dates & Assignments & Journals Remaining\\
\hline
1 & 1/11 - 1/15 & & 14 \\
\hline
2 & 1/18 - 1/22 & & 12 \\
\hline
3 & 1/25 - 1/29 & & 10 \\
\hline
4 & 2/1 - 2/5 & First Paper Due. & 8\\
\hline
5 & 2/8 - 2/12 &  & 6\\
\hline
6 & 2/15 - 2/19 &  & 4\\
\hline
7 & 2/22 - 2/26 &  Second Paper Due. & 2 \\
\hline
8 & 2/29 - 3/4 &  & 0 \\
\hline 
SPRING BREAK & 3/7 - 3/11 &  & 16 \\
\hline
9 & 3/14 - 3/18 &  & 14\\
\hline
10 EASTER BREAK (Fr)& 3/21 - 3/24 & & 12 \\
\hline
11 EASTER BREAK (Mo)& 3/29 - 4/1 & & 10 \\
\hline
12 & 4/4 - 4/8 &  Third Paper Due. & 8 \\
\hline
13 & 4/11 - 4/15 &  & 6 \\
\hline
14 & 4/18 - 4/22 &  & 4 \\
\hline
15 & 4/25 - 4/29 & Final Paper Due. & 2\\ 
\hline
16 & 5/2 - 5/4 & & 0 \\
\hline
Final's Week & 5/10 (3:00-6:00pm) & Presentations.  &\\ 
\hline
\end{tabular}
\end{center}

\subsection{Course Engagement Expectations}

The weekly workload for this course will vary by student but on average should be about 12-13 hours per week.  The follow tables provides a rough estimate of the distribution of this time over different course components for a 15 week semester. 
\begin{center}
\begin{tabular}{|l|l|l|}
\hline
In-Class &      & 3 hours/week \\ 
Finding and working with resources &        & 4 hours/week \\
Journal &   & 2.5 hours/week \\ 
Papers & 48 hours & 3.2 hours/week \\
\hline 
& & 12.7 hours/week \\ 
\hline
\end{tabular}
\end{center}

    
\end{document}