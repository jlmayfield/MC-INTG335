\documentclass[]{tufte-handout}
\usepackage{amsmath}

\title{Syllabus: INGT 388 - Artificial Intelligence and the Singularity}
\author{James Logan Mayfield }
\date{ Spring 2015 }

\begin{document}
\maketitle

\section{Logistics}
\begin{itemize}
\item \textbf{Where: } Center for Science and Business, Room 287
\item \textbf{When: } MWF 10:00 - 10:50 am
\item \textbf{Instructor :} James \textit{Logan} Mayfield
\begin{itemize}
\item \textit{Office: } Center for Science and Business, Room 344
\item \textit{Phone: } 309-457-2200
\item \textit{Email: } lmayfield \textit{at} monmouthcollege \textit{dot} edu
\item \textit{Office Hours: } By Appointment*
\end{itemize}
\item \textbf{Credits: } 1 course credit
\end{itemize}

\section{Overview}

According to some researchers, we are on the verge of a technological singularity, a moment in time where computational artificial intelligence (AI) exceeds the capabilities of natural human intelligence.  AI based technologies have already demonstrated their ability to challenge human intelligence at games of skill and knowledge such as checkers, chess and Jeopardy. They've also begun to exceed our ability to identify and discover meaningful patterns and information from data in order to perform diagnostic and predictive tasks.  What is perhaps more surprising is that AI has also begun to produce works of art comparable to those of human artists. In this class, we'll explore the myths and realities of intelligent machines and in doing so address questions about the origins, uniqueness, and prominence of human intelligence in the universe.     

We'll focus our efforts by exploring four broad topics:
\begin{enumerate}
\item When machines can and cannot out perform humans at intellectual and creative endeavours.
\item Human-Machine symbiosis. 
\item The emotional and spiritual machine
\item The artificial and the virtual: beyond intelligence.
\end{enumerate}
  

\section{Sources}

One goal of this course is for the students to direct much of the class' attention within each of the four broad topics. Towards this end, students will identify and choose resources that interest them most. The class will then decide on a few key sources to explore in depth. This means that the source list for the course is dynamic and will change as the semester progresses.

The current list of actual and potential sources is maintained by both students and the instructor as a Zotero group library and is viewable at the following URL:
\vspace{.2in}

\begin{footnotesize}
\url{https://www.zotero.org/groups/monmouth_college_-_intg-388-01_sp15}
\end{footnotesize}

 
\section{Assignments}

The course assignment workload is as follows:
\begin{itemize}
\item Reflective Writing Journal. 
\item Regular Zotero group library contributions
\item Regular participation in class discussions
\item 3 750-1250 word papers
\item 1 1500-2000 word paper
\item 1 Presentation 
\end{itemize}
The four papers will correspond to the four themes of the course. The presentation will take place during the finals period and will give you an opportunity to present something we didn't cover in the class or to go deeper on something we did cover.  

\subsection{Reflective Writing Journal}

Your journal is made up of twice weekly entries of 250-500 words each. One entry should present an observation and analysis of that observation. The other entry should consist of drawing conclusions as to the implications and meanings of a previously made observation.  Objects of observation should be drawn from or inspired by class materials or discussions.  When not obvious, be certain to clearly connect your observations to the class. Entries should not rehash class discussions. When class discussions are the object you're considering, then you should go above and beyond what was discussed in class. 

Each entry must be typed, printed and kept in an organized binder or folder.  This folder will be submitted every two weeks for grading. Each entry will be graded on a three point scale. A zero means the entry is missing. A one means the entry lacks depth of thought, lacks focus, is off topic, or has writing problems that prevent the reader from following along. A two means the writing is generally on topic, focused, and well-written, but could go deeper.  A three means the writing is very well done.  The emphasis here is the thoughts and ideas expressed and not the writing mechanics.  Journal entries have a lot more leeway than papers when it comes to writing mechanics. 

The final journal score is the average of all 32 of your individual entry scores.  Letter grades are then set as follows: 

\vspace{.1in}
\begin{center}
\begin{small}
\begin{tabular}{ll}
Entry Avg. (Min) & Letter Grade \\ \hline
2.8   & A  \\
2.75    & A- \\
2.5 & B+ \\
2.25    & B  \\ 
2   & B- \\
1.75    & C+ \\
1.5 & C  \\
1   & C- \\
0.75    & D  \\
0.5  & F 
\end{tabular}
\end{small}
\end{center}
\vspace{.1in}

Numerical values for each grade are the median value for the letter grade on the standard grading scale given above. For example, an A ranges from 94 to 100, so the median score for an A is $97.5$. 

\subsection{Zotero Library}

There is an over-abundance of sources discussing AI and its current and potential impact on our lives. Thus, we'll be developing and maintaining a set of curated resources on the areas of AI that interest \textit{us} the most using the research source management tool Zotero\sidenote{\url{https://www.zotero.org}}.  

Your individual contribution to the group library counts towards the participation component of your grade. Each student is expected to make \textit{quality} contributions to the group library. Choose resources that really interest you, are credible, and have enough intellectual meat in them for us to sink our teeth into.  For each item you add to the library, you must at least add a note that lists the following: your name, and what questions of meaning and value are raised by the resource. You are also encouraged to further develop the source within Zotero by developing tags and identifying related sources. In addition to adding your own sources, you are expected to help improve on existing resources added by your peers.


\subsection{Participation \& Attendance}

This class does not work if we're not \textit{all} freely presenting and discussing ideas.  This means you must be in class, willing to work through and with another student's ideas, and willing to present your own ideas.  \textit{You are allowed \textbf{one} unexcused absence from class.} Excused absences must be discussed with the instructor prior to the class missed.  Failure to do so is likely to result in an unexcused absence.  Every unexcused absence after the first will result in the loss of $1/5$ of your participation grade until your participation grade drops to zero after five unexcused absences. This means five or more unexcused absences caps your course grade at a maximum of a C. If you're unsure what constitutes an approved, excused, and unexcused absences, then you should check the Scots guide for the college absence policy. 

When you're in class, you're expected to lend your ideas to the discussion.  This can come in one of two forms: moving discussion to some place new by offering up something from your journal or contributing to an ongoing discussion that began from someone else's journal or the instructor. The frequency with which you do these two things will be recorded.  Your expected to offer new ideas from your journal at least once and contribute to ongoing discussions on a near daily basis. 

Assuming good attendance, your overall attendance and participation grade is a function of your contributions to in class discussions and the online Zotero group library.  There is not a set grading rubric for these things, but regular feedback and will be given through comments in journal submissions and notes on the Zotero library.  In general, if you're making thoughtful comments in class and thoughtful contributions to the Zotero library, then you're probably going to get full participation credit. If you're at all in doubt about your participation grade, do not hesitate to contact the instructor.  


\section{Grading}

This courses uses a standard grading scale.  Assignments and final grades will not be curved except in rare cases when its deemed necessary by the instruction.  Percentage grades translate to letter grades as follows:
\newline
\begin{center}
\begin{small}
\begin{tabular}{ll}
Score Range & Grade \\ \hline
94-100 & A \\
90-93 & A- \\
88-89 & B+ \\
82-87 & B \\
80-81 & B- \\
78-79 & C+ \\
72-77 & C \\
70-71 & C- \\
68-69 & D+ \\
62-67 & D \\
60-61 & D- \\
0-59 & F 
\end{tabular}
\end{small}
\end{center}


You are always welcome to challenge a grade that you feel is unfair or calculated incorrectly.  Mistakes made in your favor will never be corrected to lower your grade.  Mistakes made not in your favor will be corrected.  \textit{Basically, after the initial grading your score can only go up as the result of a challenge.}


\subsection{Grade Weights}

Your final grade is based on a weighted average of particular assignment categories.  You should be able to estimate your current grade based on your scores and these weights.  You may always visit the instructor \textit{outside of class time} to discuss your current standing.  

\begin{center}
\begin{tabular}{ll}
Category & Weight  \\ \hline
Participation \& Attendance & 25\% \\
Journal & 25\% \\
First Paper & 5\% \\
Second Paper & 10\% \\
Third Paper & 15\% \\
Final Paper & 15\% \\
Presentation & 5\% \\

\end{tabular}
\end{center}


\section{Calendar}

The following calendar should give you a feel for how work is distributed throughout the semester.  \textit{This calendar is subject to change based on the circumstances of the course.}

\begin{center}
\begin{tabular}{|c|c|r|}
\hline 
Week & Dates & Assignments \\
\hline
1 & 1/12 - 1/16 &  \\
\hline
2 & 1/19 - 1/23 & Journal Due. \\
\hline
3 & 1/26 - 1/30 &   \\
\hline
4 & 2/2 - 2/6 & Journal Due.  \\
\hline
5 & 2/9 - 2/13 &  First Paper Due.\\
\hline
6 & 2/16 - 2/20 & Journal Due. \\
\hline
7 & 2/23 - 2/27 &   \\
\hline
8 & 3/2 - 3/6 & Journal Due. Second Paper Due.  \\
\hline 
SPRING BREAK & 3/9 - 3/13 &  \\
\hline
9 & 3/16 - 3/20 &  \\
\hline
10 & 3/23 - 3/27 & Journal Due. \\
\hline
11 & 3/30 - 4/3 &  EASTER BREAK (Friday).\\
\hline
12 & 4/6 - 4/10 & EASTER BREAK (Monday). Journal Due. \\
\hline
13 & 4/13 - 4/17 &   Third Paper Due.\\
\hline
14 & 4/20 - 4/24 &  Journal Due. \\
\hline
15 & 4/27 - 5/1 &  \\ 
\hline
16 & 5/4 - 5/6 & Journal Due. Final Paper Due.\\
\hline
Final's Week & 5/12 (6:30-9:30pm) & Presentations. \\ 
\hline
\end{tabular}
\end{center}

\subsection{Course Engagement Expectations}

The weekly workload for this course will vary by student but on average should be about 12-13 hours per week.  The follow tables provides a rough estimate of the distribution of this time over different course components for a 15 week semester. 
\begin{center}
\begin{tabular}{|l|l|l|}
\hline
In-Class &      & 3 hours/week \\ 
Finding and working with resources &        & 4 hours/week \\
Journal &   & 2.5 hours/week \\ 
Papers & 48 hours & 3.2 hours/week \\
\hline 
& & 12.7 hours/week \\ 
\hline
\end{tabular}
\end{center}

    
\end{document}