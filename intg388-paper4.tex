\documentclass[]{tufte-handout}

\title{INTG 388 - Paper \#4 }
\author{}
\date{ Spring 2016 }

\begin{document}
\maketitle

\section{Prompt}

For your final paper you will use Philip K Dick's novel\citep{dick_androids_2008} and McGrath's article\citep{mcgrath_religion_2011} as guides into the quagmire that is consciousness and strong AI. For this paper you need to reflect on the problems and promises for strong AI or machine persons. Is the pursuit of strong AI the right thing for humanity? If so, what is the right path forward towards strong AI? If not, why not? What would the existence of strong AI mean \textit{for you personally?} What if we could prove strong AI were impossible, what would that mean to you? As always keep your reflection grounded in specific observations and analysis pulled from your sources and your own experiences. When seeking out additional sources, make efforts to balance the ideas of P.K.D. and McGrath with alternative views and ideas. Note that this paper is longer than your previous papers. This should allow you to further develop your ideas with such a notoriously tricky subject.  

\section{Requirements \& Logistics}

\begin{tabular}{ll}
\textsc{Due} & \textbf{Electronically via turnitin.com} on April 29  \\
\textsc{Format} & Typed. Double-spaced. \\
\textsc{Length} & \textbf{1500-2000 words} \\
\textsc{Minimum Required External Sources} & 4 (Book,Paper,+2)
\end{tabular}


\vspace{.1in}

Detailed instructions for electronic submission will be handed out in class. Grades are based on the quality and clarity of the writing, the appropriateness and quality of observations, the depth and sufficiency of the analysis, and the clarity and validity of the conclusions with respect to the overall reflections.  

\bibliographystyle{plainnat}
\bibliography{intg388-end.bib}
\end{document}	