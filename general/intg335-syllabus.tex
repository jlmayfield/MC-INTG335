\documentclass[nobib]{tufte-handout}
\usepackage{amsmath}

\title{Syllabus \\ INGT 335 --- Artificial Intelligence}
\author{}
\date{ Fall 2017 }

\begin{document}
\maketitle

\section{Logistics}
\begin{itemize}
\item \textbf{Where: } Center for Science and Business, Room 287
\item \textbf{When: } MWF 9:00--9:50 am
\item \textbf{Instructor: } James \textit{Logan} Mayfield
\begin{itemize}
\item \textit{Office: } Center for Science and Business, Room 344
\item \textit{Phone: } 309-457-2200 %chktex 8
\item \textit{Email: } lmayfield \textit{at} monmouthcollege \textit{dot} edu
\item \textit{Website: } \url{https://jlmayfield.github.io/}
\item \textit{Office Hours: } \item \textit{Office Hours: } Tuesday \& Thursday 9--10am. Friday 1--2pm. By Appointment.
\end{itemize}
\item \textbf{Website: } \url{https://jlmayfield.github.io/teaching/INTG335/}
\item \textbf{Credits: } 1 course credit
\end{itemize}

\section{Overview}

Technology based on Artificial Intelligence (AI) has already demonstrated its ability to challenge human intelligence at games of skill and knowledge such as checkers, chess and Jeopardy. It has also begun to identify and discover meaningful patterns and information from data in order to perform diagnostic and predictive tasks better than human experts.  What is perhaps more surprising is that AI has also begun to produce works of art comparable to those of human artists. This class explores the myths and realities of intelligent machines and in doing so address questions about the origins, uniqueness, and prominence of human intelligence in the universe.

The class will explore four broad topics:
\begin{enumerate}
\item Weak AI at Play
\item Weak AI at Work
\item Feeling and Falling for Weak AI
\item Strong AI
\end{enumerate}

\section{Sources}

\noindent
[1]Blaise Aguera y Arcas. 2016. Art in the Age of Machine Intelligence. Artists and Machine Intelligence. Retrieved May 17, 2017 from \url{https://medium.com/artists-and-machine-intelligence/what-is-ami-ccd936394a83}
\newline \vspace{.15in}
\noindent
[2]Brian Christian. 2011. Mind vs. Machine. The Atlantic. Retrieved May 17, 2017 from \url{https://www.theatlantic.com/magazine/archive/2011/03/mind-vs-machine/308386/}
\newline \vspace{.15in}
\noindent
[3]Philip K. Dick. 2008. Do Androids Dream of Electric Sheep? Ballantine Books.
\newline \vspace{.15in}
\noindent
[4]Frank Marshall. 2015. The Man vs. The Machine. Retrieved January 21, 2015 from \url{http://espn.go.com/video/clip?id=11694550}
\newline \vspace{.15in}
\noindent
[5]Jad Abumrad and Robert Krulwich. Talking to Machines. Retrieved January 3, 2015 from \url{http://www.radiolab.org/story/137407-talking-to-machines/}
\newline \vspace{.15in}
\noindent
[6]James F. McGrath. 2011. Robots, Rights and Religion. In Religion and Science Fiction, James F. McGrath (ed.). Wipf \& Stock Pub, 42.
\newline \vspace{.15in}
\noindent
[7]M. Mori, K.F. MacDorman, and N. Kageki. 2012. The Uncanny Valley. IEEE Robotics Automation Magazine 19, 2 (June 2012), 98--100.
\newline \vspace{.15in}
\noindent
[8]Casey Newton. 2016. When her best friend died, she used artificial intelligence to keep talking to him. TheVerge.com. Retrieved May 17, 2017 from \url{http://www.theverge.com/a/luka-artificial-intelligence-memorial-roman-mazurenko-bot}
\newline \vspace{.15in}
\noindent
[9]Planet Money. Humans vs. Robots. Retrieved May 19, 2015 from \url{http://www.npr.org/sections/money/2015/05/08/405270046/episode-622-humans-vs-robots}
\newline \vspace{.15in}
\noindent
[10]Planet Money. The Machine Comes To Town. Retrieved May 19, 2015 from \url{http://www.npr.org/sections/money/2015/05/13/406461675/episode-623-the-machine-comes-to-town}
\newline \vspace{.15in}
\noindent
[11]Planet Money. The People Inside Your Machine. Retrieved February 22, 2015 from \url{http://www.npr.org/blogs/money/2015/01/30/382657657/episode-600-the-people-inside-your-machine}
\newline \vspace{.15in}
\noindent
[12]Planet Money. When Luddites Attack. Retrieved May 19, 2015 from \url{http://www.npr.org/sections/money/2015/05/06/404701816/episode-621-when-luddites-attack}
\newline \vspace{.15in}
\noindent
[13]Stuart Jonathan Russell and Peter Norvig. 2010. Introduction. In Artificial Intelligence: A Modern Approach (Third). Prentice Hall, 1--31.
\newline \vspace{.15in}
\noindent
[14]Stuart Jonathan Russell and Peter Norvig. 2010. Philosophical Foundations. In Artificial Intelligence: A Modern Approach (Third). Prentice Hall, 1020--1042.
\newline \vspace{.15in}
\noindent
[15]Samuel Hansen. Chinook. Retrieved January 2, 2015 from \url{http://relprime.com/chinook/}
\newline \vspace{.15in}
\noindent
[16]Jake Schreier. 2012. Robot \& Frank. Stage 6 Films.
\newline \vspace{.15in}
\noindent
[17]Nate Silver. 2015. Rage Against The Machines. In The Signal and the Noise: Why So Many Predictions Fail--but Some Don't. Penguin Books. Retrieved from \url{https://fivethirtyeight.com/features/rage-against-the-machines/}
\newline \vspace{.15in}
\noindent
[18]John Supko. 2015. How I Taught My Computer to Write Its Own Music. Nautilus. Retrieved February 12, 2015 from \url{http://nautil.us/issue/21/information/how-i-taught-my-computer-to-write-its-own-music}
\newline \vspace{.15in}
\noindent
[19]2016. Answering the machinery question. Economist 419, 8995 (June 2016), 15--16.
\newline \vspace{.15in}
\noindent
[20]2016. Frankenstein's paperclips. Economist 419, 8995 (June 2016), 13--15.
\newline \vspace{.15in}
\noindent
[21]2016. Re-educating Rita. Economist 419, 8995 (June 2016), 10--12.
\newline \vspace{.15in}
\noindent
[22]2016. The return of the machinery question. Economist 419, 8995 (June 2016), 1--4.


\section{Assignments}

The course assignment workload is as follows:
\begin{itemize}
\item Regular participation in class discussions
\item 10--12 Short Journal Writings
\item 3 750--1250 word papers
\item 1 1500--2000 word paper
\item 1 Final In-Class writing.
\end{itemize}
The four papers will correspond to the four themes of the course. Journals are short writings that are periodically assigned between papers. They are meant to help you catalyze your thinking leading up to a paper. The final will include a single reflective writing assignment.

\subsection{Participation \& Attendance}

This class does not work if students are not \textit{all} freely presenting and discussing ideas.  This means each student must be in class, willing to work through and with another student's ideas, and willing to present their own ideas.  \textit{A student is allowed \textbf{one} unexcused absence from class.} Excused absences must be discussed with the instructor prior to the class missed.  Failure to do so is likely to result in an unexcused absence.  Every unexcused absence after the first will result in the loss of \(1/5\) of the course participation grade until the participation grade drops to zero after five unexcused absences. This means five or more unexcused absences caps the student's course grade at a maximum of a C. What constitutes an approved, excused, and unexcused absence is well documented in the Scots guide under the college absence policy.

Assuming good attendance, the overall attendance and participation grade is a function of a student's contributions to in class discussion.  Discussions will occur in small groups and as a whole class. Students should contribute in both cases.


\subsection{The Writing Center}

The Writing Center offers unlimited, free peer tutoring sessions for Monmouth College students.  Peer writing tutors are trained to work with writers from any major, of any writing ability, on any type of writing assignment, and at any stage of their writing processes, from planning to drafting to revising to editing.  Peer speech tutors are also available on a limited basis to assist student speakers at any point in the process of designing a speech --- from outlining to delivery.  The Writing Center is  located on the 3rd floor of the Mellinger Teaching and Learning Center, and is open Sunday-Thursday 7--10pm and Monday-Thursday 3--5pm on a first-come, first-served basis.  No appointment necessary!

\section{Grading}

This courses uses a standard grading scale.  Assignments and final grades will not be curved except in rare cases when its deemed necessary by the instruction.  Percentage grades translate to letter grades as follows:
\newline
\begin{center}
\begin{small}
\begin{tabular}{ll}
\underline{Score Range} & \underline{Grade} \\
94--100 & A \\
90--93 & A- \\
88--89 & B+ \\
82--87 & B \\
80--81 & B- \\
78--79 & C+ \\
72--77 & C \\
70--71 & C- \\
68--69 & D+ \\
62--67 & D \\
60--61 & D- \\
0--59 & F
\end{tabular}
\end{small}
\end{center}


Students are always welcome to challenge a grade that they feel is unfair or calculated incorrectly.  Mistakes made in the student's favor will never be corrected to lower a grade.  Mistakes made not in a student's favor will be corrected.  \textit{Basically, after the initial grading of an assignment, it's score can only go up as the result of a challenge.}


\subsection{Grade Weights}

The final grade is based on a weighted average of particular assignment categories.  Students may visit the instructor \textit{outside of class time} to discuss their current standing.

\begin{center}
\begin{tabular}{ll}
\underline{Category} & \underline{Weight}  \\
Participation \& Attendance & 25\% \\
Journals & 30\% \\
First Paper & 5\% \\
Second Paper & 10\% \\
Third Paper & 10\% \\
Fourth Paper & 15\% \\
Final & 5\% \\
\end{tabular}
\end{center}


\section{Calendar}

\textit{This calendar is subject to change based on the circumstances of the course.}

\begin{center}
\begin{tabular}{llll}
\underline{Week} & \underline{Dates} & \underline{Assignments Due} & \underline{Sources} \\
1 & 8/22 --- 8/25 & &  \\
2 & 8/28 --- 9/1 & Journal 1 (M), 2 (F).  &  13,14 \\
3 & 9/4 --- 9/8 & Journal 3 (W). & 17,4,15 \\
4 & 9/11 --- 9/15 & Journal 4 (M) & 12,9,10 \\
5 & 9/18 --- 9/22 & Paper 1 (M). Journal 5(F) & 11,21,19,22 \\
6 & 9/25 --- 9/29 & Journal 6 (W). &  1,18 \\
7 & 10/2 --- 10/6 & Journal 7 (M). & 5 \\
8 & 10/9 --- 10/10 & Paper 2 (M). FALL BREAK (WF) & 2 \\
9 & 10/16 --- 10/20 & Journal 8 (M). & 8,4,7 \\
10 & 10/23 --- 10/27 & Jouranl 9 (M). & 20,6 \\
11 & 10/30 --- 11/3 & Paper 3 (M). Journal 10 (W). & 6,3(1--3) \\
12 & 11/6 --- 11/10 & Journal 11 (M). & 3(4--5),3(6--8),3(9--11) \\
13 & 11/13 --- 11/17 & Journal 12(M) &  3(12--15),3(16--17),3(18--19) \\
14 & 11/20 --- 11/21 & Journal 13 (M). TDAY BREAK (WF) & 3(20--22)\\
15 & 11/27 --- 12/1 & Journal 14 (W)  & \\
16 & 12/4 --- 12/6 & Paper 4 (M)  & \\
Final's Week & 12/11 (6:30pm--9:30pm) &  &  \\
\end{tabular}
\end{center}


\subsection{Course Engagement Expectations}

The weekly workload for this course will vary by student but on average should be about 12--13 hours per week.  The follow tables provides a rough estimate of the distribution of this time over different course components for a 15 week semester.

\begin{center}
\begin{tabular}{lll}
In-Class &      & 3 hours/week \\
Working with Sources &        & 4 hours/week \\
Journals &   & 2 hours/week \\
Papers &  & 3 hours/week \\

& & 12 hours/week \\

\end{tabular}
\end{center}

\end{document}
