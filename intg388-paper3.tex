\documentclass[nobib]{tufte-handout}

\title{INTG 388 - Paper \#3 }
\author{}
\date{ Spring 2016 }

\begin{document}
\maketitle

\section{Prompt}


In this your third paper you'll use AI as a vehicle to explore your rational and emotional self.  Let's continue under the premise that all current AI is weak AI and that the primary purpose of AI is to be used as a tool to satisfy some underlying need of it's user. In your last paper you explored the fact that by allowing some of our needs to be satisfied by AI, we are, at least in part, giving control over some portion of our lives to the AI.  For this paper you'll look more closely at the source and nature of that need. 

We've now seen AI used as a tool for both analytical, rational intelligence and emotional intelligence. For this reflection I want you to compare and contrast your reactions and feelings regarding the use of AI for satisfying analytical needs versus emotional needs.  In doing so, you should explore the extent to which you value your own analytical, rational intelligence and your emotional intelligence. Examine how your feelings and reactions to AI from both domains reflect your rational and emotional self. More so than any previous paper, this paper should focus clearly on what our reactions and attitudes towards AI tell us about ourselves and not just about the AI and what it does. 

\section{Requirements \& Logistics}

\begin{tabular}{ll}
\textsc{Due} & In Class on April 1st  \\
\textsc{Format} & Typed. Printed. Double-spaced. Stapled. \\
\textsc{Length} & 750 - 1250 words \\
\textsc{Minimum Required External Sources} & 3 minimum.  
\end{tabular}

\vspace{.1in}
Grades are based on the quality and clarity of the writing, the appropriateness and quality of observations, the depth and sufficiency of the analysis, and the clarity and validity of the conclusions with respect to the overall reflections.  


\end{document}	