\documentclass[]{tufte-handout}

\title{INTG 388 - Paper \#3 }
\author{James Logan Mayfield}
\date{ Spring 2015 }

\begin{document}
\maketitle

\section{Prompt}


In this your third paper you'll use AI as a vehicle to explore your rational and emotional self.  Let's continue under the premise that all current AI is weak AI and that the primary purpose of AI is to be used as a tool to satisfy some underlying need of it's user. In your last paper you explored the fact that by allowing some of our needs to be satisfied by AI, we are, at least in part, giving control over some portion of our lives to the AI.  For this paper you'll look more closely at the source and nature of that need. 

We've now seen AI used as a tool for both analytical, rational intelligence and emotional intelligence. For this reflection I want you to compare and contrast your reactions and feelings regarding the use of AI for satisfying analytical needs versus emotional needs.  In doing so, you should explore the extent to which you value your own analytical, rational intelligence and your emotional intelligence. Examine how your feelings and reactions to AI from both domains reflect your rational and emotional self. 

\section{Requirements \& Logistics}

\begin{tabular}{ll}
\textsc{Due} & In Class on April 13  \\
\textsc{Format} & Typed. Printed. Double-spaced. Stapled. \\
\textsc{Length} & 750 - 1250 words \\
\textsc{Minimum Required External Sources} & 3 minimum.  
\end{tabular}

\vspace{.1in}
Grades are based on the quality and clarity of the writing, the appropriateness and quality of observations, the depth and sufficiency of the analysis, and the clarity and validity of the conclusions with respect to the overall reflections.  

\section{Reflective Writing}

Reflection is an inward looking process. We're trying to explore what's inside us as individuals and a species. This paper is not about making a point or presenting information. It's about exploring an experience, and idea, or a possible answer to a questions and seeing how it impacts us on a personal level. We're exploring the human condition and how we make sense of it. We can think of this process in terms of three phases:
\begin{enumerate}
\item \textbf{Observation:} Students should carefully describe, explain, or notice. Their writing should make clear that they've taken some time to carefully observe. \newline
\item \textbf{Analysis:} After observing, students should then explore. They should break ideas down and put them back together. They should begin making connections between ideas encountered in the course and their own experience. They should consider what their reflection means to them intellectually, personally, and socially. \newline
\item \textbf{Deriving Implications and Meaning:} Students should offer conclusions about where the reflection has brought them. These conclusions may be partial, plural, or not fully settled, but students should be able to articulate how they think differently at the end of their reflection. They may have new ideas. They may have new understandings of old beliefs. In any case, they should articulate where they stand in relation to the reflection practiced. 
\end{enumerate}

\section{The Writing Center}

Wonderful ideas and deep reflection can be lost or ruined by muddled writing.  I grew most as a writer when I cared deeply about the things I was writing and didn't want there to be any uncertainty on the part of my readers. Perhaps more than other forms or writing, reflective writing is ultimately a reflection of ourselves. You should want to put your best word forward when it's you on the page. Writing well is hard. It takes practice. I requires help. I encourage \textit{all} of you to take this opportunity to step up your writing game. The writing center is here to help you. They're awesome. Don't wait until my grade to get feedback on the quality and clarity of your writing.  
\begin{quote}
The Monmouth College Writing Center offers unlimited, free peer tutoring sessions for students at MC.  Peer writing tutors work with writers from any major, of any writing ability, on any type of writing assignment, and at any stage of their writing processes, from planning to drafting to revising to editing.  We are located on the 3rd floor of the Mellinger Teaching and Learning Center, and we are open Sunday-Thursday 7-10pm and Monday-Thursday 3-5pm on a first-come, first-served basis.  No appointment necessary!  International students can sign up to work with the same writing tutor for a weekly session. Learn more about the Writing Center at our website: \url{http://blogs.monm.edu/writingatmc/writing-center/}
\end{quote}



\end{document}	