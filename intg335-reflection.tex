\documentclass[nobib]{tufte-handout}
\usepackage{amsmath}
\usepackage{setspace}
\usepackage{hyperref}

\title{INTG 335 \\ On Reflection and Reflective Writing}
\author{James Logan Mayfield }
\date{ Fall 2017 }


\begin{document}
\maketitle
\thispagestyle{empty}

\subsection*{Normative vs. Positive Statements}

Unless you're a student of ethics, philosophy, or religious studies, then you're probably used to working in the world of  \textit{positive statements}. Such statements describe what is and how it is.  Looking at a piece of literature or art and describing the methodology used to create the work is certainly done with positive statements. Similarly, offering up an interpretation of the work and perhaps the authors intent is in this same vane. Science and Economics are certainly focused on describe what is in the world and how it works. This mode of objective critical thought and analysis is not what we're after in this course. We wish to explore something more subjective.

A \textit{normative statement} is focused on what should be rather than what is. It emphasizes value and meaning.  If when discussion a work of art you are addressing the aesthetics of that work, then you're likely to be speaking with a normative voice. If a scientist is arguing against the pursuit of a specific technology due to ethical concerns, then she is working with normative material as well.  Such works are tricky as meaning and value in this sense are not universally agreed upon. It varies from culture to culture, group to group, and individual to individual.

The goal of this course is to explore these tricky questions of meaning and value. Rather than attempting to speak for the values of ``society'', or ``people'', we'll focus squarely on the one slice of humanity that we are the definitive experts on: ourselves. Our goal is not merely to explore that which is meaningful and valuable, but that which is meaningful and valuable to us. In doing so we're likely to find that even within a group the size of this class there exists differing opinions on what should and should not be in the world. For this reason we must be ready to maintain a critical and analytical mind. We must explore the sources, merits, and implications of our own values while trying to understand those of others. With this done we can consider the question of how to coexist in the face of such diversity.

\subsection*{What is Reflection?}

Reflection is an inward looking process intent on self-discovery. It's about exploring an experience, an idea, or a discovery, seeing what impact it has had on us as an individual, understanding what that means for us moving forward, and ultimately how we have changed as a result of such deep introspection. In this class we'll look at this process in terms of three phases\sidenote{This rubric and much the language describing it was developed by the Reflections coordinatorsand teachers at Monmouth College.}:
\begin{enumerate}
\item \textbf{Observation:} Carefully describe, explain, or notice. Take some time time to carefully observe a thing. This is the experience, the idea, or the discovery. \newline

\item \textbf{Analysis:} After observing, explore. Break ideas down and put them back together. Make connections between ideas encountered in the course and your own experience. Consider what your reflection means to you intellectually, personally, and socially.  As always, your analysis should be careful and critical. The new element is that it should also be personal.  \newline

\item \textbf{Deriving Implications and Meaning:} Offer conclusions about where the reflection has brought you. These conclusions may be partial, plural, or not fully settled, but you should be able to articulate how you think differently at the end of their reflection. You may have new ideas.An You may have new understandings of old beliefs. In any case, You should articulate where you stand in relation to the reflection practiced.
\end{enumerate}

Remember our first goal is to gain a deeper understanding of our own system of values and what we find most meaningful in our lives. We cannot engage in the larger, societal discussion if we don't understand our personal values and how to articulate those to others. For this reason we always keep the subject of our reflection fixed: our personal values, ethics, and aesthetics. When we look at the values of others it is to contextualize our own feelings, to understand how they fit in wider spectrum, and how acknowledging that spectrum effects us and our sense of
meaning.

Good reflections begin with objects and experiences that offer some level diversity. We must avoid building echo chambers out of observations of things that are familiar and comfortable.  When the objects of our reflection challenge our previously held beliefs, even in smalls ways, then we have something to analyze in depth. We have contrast and from that contrast we can paint a more nuanced picture of what we value. Your goal isn't to prove yourself or anyone else wrong, it's simply to understand what's important to you with more depth and clarity than you have before.


\subsection*{How to write about Reflection}

The first step is to do the reflection in whatever way suits you. Think it out in your head. Jot down notes. Write down ideas. Use whatever process you need to use in order to get though the reflection. Once that's done and you can clearly identify what you've observed, the ways in which you've analyzed it, and have a clearly articulated conclusion to the reflection, then you can engage in the process of writing this down and presenting it to others. That is to say that the paper is a presentation of the reflection and not the reflection itself.

The style of reflective writing I advocate begins by treating your conclusion as the ``thesis''. It is thing you're presenting, explaining, and developing in the course of your writing. Like any other thesis, it should be clearly stated in the introductory paragraph and then developed and expanded upon in the remainder of the document. Too often I see students carry out the phases of reflection in a linear fashion in their writing then submit that as their reflective writing.  They observer, then analysis, then conclude.  The end result, from the reader's perspective, is usually confusing. It's not always clear where you're going with this and why we should keep reading. You, the reflector, should absolutely observer, analyze, then conclude, but you should do that \textit{before you write the paper}.  Always remember, the goal of the paper is to clearly communicate the conclusions you reached as a result of your reflection. The later comes before the former.

Your goal as the writer is to make sure the reader understands the path that lead you to your new way of thinking. This is why you should present the conclusion right up front. The reader will begin their journey with a broad understanding of  where your thinking was before the reflection, the object of your reflection, and the new way of thinking that resulted from the reflection. You then fill in the details and conclude the essay by going back to the conclusion of the reflection with the details fully explored. \textit{You are 100\% not trying to convince them that your new way of thinking is \textit{the} way to think on the subject nor are you informing them of new facts.} The subject the reader should learn the most about from reading your essay is you. Their thinking may change or they may learn some new facts and ideas as a result of reading your essay, but that's secondary to the real goal of presenting and explaining your change in thinking and the reflection that got you there.


\end{document}
