\documentclass[nobib]{tufte-handout}

\title{INGT 335 --- Artificial Intelligence \\ Journal 9}
\author{}
\date{ Fall 2017 }

\begin{document}
\maketitle

\begin{center}
  \textbf{Due Monday 10/23}
\end{center}

Robot from \textit{Robot and Frank} could lie.  It did so in order to manipulate Frank into leading a healthier lifestyle and doing what he needed to do to manage his dementia.  The implication here is that either by design or by learned behavior, the Robot has determined that the best way to deliver health care requires the occasional lie. Is lying a necessary evil? Would you be OK with an AI lying to you means they're better at what they do? What if a human lies to you for the same purpose? Reflect on the moral and rational conflict that surrounds lying and how you feel it should play out in the world of AI such as Robot.



\end{document}
