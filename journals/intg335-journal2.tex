\documentclass[nobib]{tufte-handout}

\title{INGT 335 --- Artificial Intelligence \\ Journal 2}
\author{}
\date{ Fall 2017 }

\begin{document}
\maketitle


For this journal you'll reflect on something from the Russel \& Norvig readings or the subsequent discussions.

\section{The Assignment}

The Russel \& Norvig readings introduced you to the reality of the pursuit of AI and the surrounding issues.  You must now choose a single, specific element of the reading as the basis for your reflection.  You should then outline the three elements of the reflection\sidenote{Observation, Analysis, \& and Conclusion} and then write a brief 1--2 paragraph presentation of the outlined reflection.

\subsection{An Example from Class}
In class I broke down the elements of a reflection for the intelligence spectrum exercise we did in class. An outline for that reflection might look like this:
  \begin{itemize}
    \item \textit{Observation} Emotional Intellience was far below Rocket Science\sidenote{Notice it's very specific. It's about two marks on the line. That's it.}
    \item \textit{Analysis} My experiences growing up, the home I grew up in, and the environment in which I work all agree with placing techinial intelligience (i.e. Rocket Science) above emotional intelligence (i.e. Leadership and inter-personal skills), and I don't think they should\sidenote{A normative statement about the value of specific forms of intelligence}.  I too often forget that the sucessfully naviagtion of human relationships and human interaction can be every bit as hard as math and science.\sidenote{Notice this is personal. It's about my values as they are exposed by others' values. Above all, it's about meaning and value, not definition and taxonomy.}
    \item \textit{Conclusion} This exercise reminded me that I need to work to keep broad my perspective on what does and does not constitute intelligence. I do myself and others a diservice when I don't credit their intelligence in all its many and varied forms.\sidenote{Again. It's personal. It's an affirmation of what I value and not a declaration of what others should value.}
  \end{itemize}
Pay special attention to the side notes. The observation should be \textit{very} specific. Less is more. The analysis and conclusions should be personal. I left out details that I told you in class: my parents comments about me relative to my sister, the educational background of my family, and the nature of my profession. These things would go in the paragraphs I'd write from this outline. They flesh out the analysis and provide context to the conclusion and in doing so help the reader to understand where I'm coming from.


\end{document}
