\documentclass[nobib]{tufte-handout}
\usepackage{amsmath}

\title{INTG 335 \\ Paper \#2 }
%\author{ James Logan Mayfield }
\date{ Fall 2017 }

\begin{document}
\maketitle

\section{Prompt}

We've turned our attention from games to tasks and skills that determine careers and various aspects of everyday life. What seems to be a recurring theme is the continued supremacy of AI\@. A self-driving car is a better driver than a human and will save lives. Automated factories are safer, more efficient, and will lead to cheaper goods. AI are even able to produce works of art comparable to human artists and can be used to means to extend the human artists capabilities. In this paper you'll push back, or not, on the rise of the machines.

%Refine Questions
Where do we draw the line? When do we accept sub-optimal results produced by humans? When should ethics or aesthetics trump efficiency and ease of use? 

\section{Requirements \& Logistics}

\begin{tabular}{ll}
\textsc{Due} &  \\
\textsc{Format} & Typed. Printed. Double-spaced. Stapled. \\
\textsc{Length} & 750--1250 words \\
\textsc{Min. Sources} & 3
\end{tabular}
\vspace{.25in}

Grades are based on the quality and clarity of the writing, the appropriateness and quality of observations, the depth and sufficiency of the analysis, and the clarity and validity of the conclusions with respect to the overall reflections. Be sure to review \textit{On Reflection and Reflective Writing}\sidenote{\url{https://jlmayfield.github.io/teaching/INTG335/intg335-reflection.pdf}} if needed.

\section{The Writing Center}

Wonderful ideas and deep reflection can be lost or ruined by muddled writing.  I grew most as a writer when I cared deeply about the things I was writing and didn't want there to be any uncertainty on the part of my readers. Perhaps more than other forms or writing, reflective writing is ultimately a reflection of ourselves. You should want to put your best word forward when it's you on the page. Writing well is hard. It takes practice. I requires help. I encourage \textit{all} of you to take this opportunity to step up your writing game. The writing center is here to help you. They're awesome. Don't wait until my grade to get feedback on the quality and clarity of your writing.
\begin{quote}
The Monmouth College Writing Center offers unlimited, free peer tutoring sessions for students at MC\@.  Peer writing tutors work with writers from any major, of any writing ability, on any type of writing assignment, and at any stage of their writing processes, from planning to drafting to revising to editing.  We are located on the 3rd floor of the Mellinger Teaching and Learning Center, and the Writing Center is open Sun-Thurs 7--10pm, and Mon-Thurs 3--5pm on a first-come, first-served basis.  No appointment necessary!  International students can sign up to work with the same writing tutor for a weekly session. Learn more about the Writing Center at our website: \url{http://blogs.monm.edu/writingatmc/writing-center/}
\end{quote}


\end{document}
